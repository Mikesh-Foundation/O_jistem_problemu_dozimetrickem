\section{Reakční rychlost}
Jak bylo zmíněno v předchozí kapitole je aktivita počet rozpadů daného radionuklidu za jednotku času. Tyto radionuklidy ale mohly vzniknout nekonečným počtem způsobů. Pro stanovení spektra neutronů, kterým byl aktivační detektor ozářen, má \gls{aktivita}, ať už vztažená na jádro nebo na hmotnost, velmi omezenou vypovídací hodnotu. Nicméně je možné pomocí aktivity stanovit reakční rychlost, která již vypovídací hodnotu má.

Reakční rychlost vyjadřuje, ke kolika reakcím daného typu dojde za jednotku času a je definována následovně
\begin{equation}
    \label{rov401}
    RR = \int\displaylimits_{V}\int\displaylimits_{0}^{\infty} \Phi(E,\vec{x}) \cdot \Sigma(E,\vec{x}) \: dE dV
\end{equation}
kde:
\begin{tabbing}
    \phantom{$D_{n50}\ $}\= \kill
    $RR$\> = Reakční rychlost v \acr{AD}. \\
    $\Phi$\> = Hustota toku neutronů. \\
    $\Sigma$\> = Makroskopický účinný průřez. \\
    $E$\> = Kinetická energie neutronu. \\
    $V$\> = Objem aktivačního detektoru. \\
    $\vec{x}$\> = Polohový vektor v objemu aktivačního detektoru. \\
\end{tabbing}

Protože v praxi se pracuje s grupovými veličinami, tak bývá rovnice (\ref{rov401}) diskretizována následujícím způsobem
\begin{equation}
    \label{rov402}
    RR = V \cdot \sum_{g=1}^{G} \Phi_g \cdot \Sigma_g
\end{equation}
kde:
\begin{tabbing}
    \phantom{$D_{n50}\ $}\= \kill
    $RR$\> = Reakční rychlost. \\
    $\Phi_g$\> = Integrální hustota toku neutronů v energetické grupě $g$. \\
    $\Sigma_g$\> = Makroskopický účinný průřez v energetické grupě $g$. \\
    $g$\> = Číslo energetické \glslink{grupa}{grupy}. \\
    $G$\> = Počet energetických \glslink{grupa}{grup}. \\
    $V$\> = Objem aktivačního detektoru. \\
\end{tabbing}

Objem $V$ v rovnici (\ref{rov402}) vyjadřuje objem celého \acr{AD} a je předpokládáno, že atomová hustota terčových jader a hustota toku neutronů je konstantní v celém tomto objemu. 

Je zřejmé, že na rozdíl od \glslink{aktivita}{aktivity} je reakční rychlost veličina, která je nenulová pouze během ozařování \acr{AD}.

\subsection{RR vztažená na počet terčových jader}
Z praktických důvodů je $RR$ často vztahována na počet jader terčového nuklidu v aktivačním detektoru před ozářením. Takto normovaná $RR$ je definovaná následovně 
\begin{equation}
    \label{rov403}
    \tilde{RR} = \frac{\int\displaylimits_{V}\int\displaylimits_{0}^{\infty} \Phi(E,\vec{x}) \cdot \sigma(E) \cdot N_t(\vec{x}) \: dE dV}{\int\displaylimits_{V}N_t(\vec{x}) \: dV}
\end{equation}
kde:
\begin{tabbing}
    \phantom{$D_{n50}\ $}\= \kill
    $\tilde{RR}$\> = Reakční rychlost vztažená na počet terčových jader v \acr{AD}. \\
    $\Phi$\> = Hustota toku neutronů. \\
    $\sigma$\> = Mikroskopický účinný průřez. \\
    $N_t$\> = Atomová hustota terčových jader v \acr{AD}. \\
    $E$\> = Kinetická energie neutronu. \\
    $V$\> = Objem aktivačního detektoru. \\
    $\vec{x}$\> = Polohový vektor v objemu \acr{AD}. \\
\end{tabbing}

\begin{equation}
    \label{rov404}
    \tilde{RR} = \frac{RR}{N_t \cdot V} = \sum_{g=1}^{G} \Phi_g \cdot \sigma_g
\end{equation}
kde:
\begin{tabbing}
    \phantom{$D_{n50}\ $}\= \kill
    $\tilde{RR}$\> = Reakční rychlost vztažená na počet terčových jader v \acr{AD}. \\
    $RR$\> = Reakční rychlost. \\
    $N_t$\> = Atomová hustota terčových jader v \acr{AD}. \\
    $V$\> = Objem aktivačního detektoru. \\
    $\Phi_g$\> = Integrální hustota toku neutronů v energetické grupě $g$. \\
    $\sigma_g$\> = Mikroskopický účinný průřez v energetické grupě $g$. \\
    $g$\> = Číslo energetické \glslink{grupa}{grupy}. \\
    $G$\> = Počet energetických \glslink{grupa}{grup}. \\
\end{tabbing}