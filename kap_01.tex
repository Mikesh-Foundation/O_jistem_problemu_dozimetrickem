\section{Úvod}
Vlivem ionizujícího záření dochází k degradaci vlastností materiálů jednotlivých komponent jaderného reaktoru, což může vést k jejich selhání. Poškození daného materiálu vlivem ionizujícího záření se odvíjí od množství, typu a energie částic, které za danou dobu interagovaly s tímto materiálem. Určení \gls{fluence} těchto částic je proto stěžejní úlohou při stanovení míry materiálové degradace. Stanovením energetické distribuce částic a jejich \gls{fluence} se zabývá vědní obor nazývaný \glslink{reaktorova dozimetrie}{reaktorová dozimetrie}.

V jaderném reaktoru jsou komponenty poškozovány primárně \glslink{rychle neutrony}{rychlými neutrony}. Jestliže lze považovat \glslink{spektrum neutronu}{spektrum} neutronů jako neměnné za určitou dobu, je hodnota \gls{fluence} neutronů za tuto dobu dána pouze tvarem tohoto spektra a časem ozařování materiálu. Známe-li tedy historii provozu sledovaného reaktoru, lze úlohu redukovat na určení \glslink{spektrum neutronu}{spektra neutronů}.

Spektrum neutronů je možné určit řadou experimentálních a výpočetních metod v závislosti na podmínkách uvnitř sledovaného reaktoru. Tento dokument se věnuje zcela výhradně experimentální metodě používající aktivační detektory. Tyto detektory mají několik výhod oproti jiným metodám určení neutronového spektra. Protože aktivační detektory mají převážně podobu kovových folií nebo drátů nevelkých rozměrů%
\footnote{Folie mívají plochu v řádu jednotek $cm^2$ a tloušťku ve zlomcích $mm$. Dráty pak průměr od desetin $mm$ do jednotek $mm$. Rozměry jsou voleny na základě  očekávaných parametrů spektra neutronů.}
, je možné je umístit do stísněných míst v reaktoru např. uvnitř \acr{AZ}. Aktivační detektory je také možné vystavit vlivu vysokých teplot a gama záření, aniž by tím byl významně ovlivněn průběh a výsledek experimentu.

Mezi relativní nevýhody tohoto typu detektorů patří fakt, že jsou citlivé v širokém rozsahu energií%
\footnote{Zde jde spíše o míru aktivace neutrony různých energií. Ve skutečnosti jsou materiály více či méně aktivovány neutrony všech energií.}
, který je dán fyzikálními vlastnostmi%
\footnote{Dobře popsaný účinný průřez je zásadní pro správné vyhodnocení měření.}
materiálu, z kterého je detektor vyroben. Změřená aktivita jednoho detektoru tak vypovídá pouze o množství neutronů v tomto rozsahu energií, ale další informaci o energetické distribuci uvnitř tohoto intervalu již neposkytne.  Jeden aktivační detektor tak podá pouze velmi omezenou informaci o tvaru \glslink{spektrum neutronu}{spektra neutronů}. Proto je třeba v rámci jednoho experimentu kombinovat více správně zvolených aktivačních detektorů z různých materiálů, abychom získali co nejvíce informací o sledovaném spektru neutronů.

Další klíčovou vlastností aktivačních detektorů je nestabilita produktů sledovaných reakcí. Tato vlastnost na jednu stranu umožňuje relativně snadné měření aktivačních detektorů, ale na straně druhé dojde po jisté době k úplné ztrátě aktivity, což znemožní další měření v budoucnu. \Gls{saturace} během ozařování je opět důsledkem této vlastnosti.
